\documentclass[9pt,landscape]{article}
\usepackage{multicol}
\usepackage{calc}
\usepackage[landscape]{geometry}
\usepackage{amsmath,amsthm,amsfonts,amssymb,relsize}
\usepackage{color,graphicx,overpic}
\usepackage{enumitem}

% Turn off header and footer
\pagestyle{empty}
\geometry{top=.12in,left=.25in,right=.25in,bottom=.12in}

% Redefine section commands to use less space
\makeatletter
\renewcommand{\section}{\@startsection{section}{1}{\z@}{2ex}{1ex}
                       {\normalfont\normalsize\bfseries\textit}}
\renewcommand{\subsection}{\@startsection{subsection}{2}{\z@}{2ex}{0.5ex}
                          {\normalfont\small\bfseries}}
\makeatother

% Don't print section numbers
\setcounter{secnumdepth}{0}

\setlength{\parindent}{0pt}
\setlength{\parskip}{0pt}

\newcommand{\R}{\mathbb{R}}
\newcommand{\E}{\mathbb{E}}

\newcommand{\bb}{\mathbf{b}}
\newcommand{\bv}{\mathbf{v}}
\newcommand{\bx}{\mathbf{x}}
\newcommand{\by}{\mathbf{y}}
\newcommand{\bz}{\mathbf{z}}

\newcommand{\bA}{\mathbf{A}}
\newcommand{\bB}{\mathbf{B}}
\newcommand{\bH}{\mathbf{H}}
\newcommand{\bI}{\mathbf{I}}
\newcommand{\bJ}{\mathbf{J}}
\newcommand{\bX}{\mathbf{X}}
\newcommand{\bY}{\mathbf{Y}}
\newcommand{\bZ}{\mathbf{Z}}

\newcommand{\balpha}{\boldsymbol{\alpha}}
\newcommand{\bbeta}{\boldsymbol{\beta}}
\newcommand{\btheta}{\boldsymbol{\theta}}
\newcommand{\bmu}{\boldsymbol{\mu}}
\newcommand{\bxi}{\boldsymbol{\xi}}
\newcommand{\bSigma}{\boldsymbol{\Sigma}}
\newcommand{\bLambda}{\boldsymbol{\Lambda}}

\newcommand{\diag}{\text{diag}}
\newcommand{\tp}[1]{#1^{\prime}}
\newcommand{\inv}[1]{#1^{-1}}
\newcommand{\norm}[1]{\|#1\|}
\newcommand{\gaussian}[2]{\mathcal{N}(#1, #2)}
\newcommand{\inR}[2]{#1 \in \R^{#2}}


\newcommand{\ruler}{\\\rule{\columnwidth}{0.25pt}\\}
\newcommand{\uruler}{\rule{\columnwidth}{0.25pt}\\}

\newcommand{\itemizebegin}{\begin{itemize}[noitemsep,topsep=0pt]}
\newcommand{\itemizeend}{\end{itemize}}

\newcommand{\enumbegin}{\begin{enumerate}[noitemsep,topsep=0pt]}
\newcommand{\enumend}{\end{enumerate}}

\DeclareMathOperator*{\argmin}{arg\,min}



% -----------------------------------------------------------------------
\begin{document}
\raggedright
\footnotesize
\begin{multicols*}{3}
% multicol parameters
% These lengths are set only within the two main columns
% \setlength{\columnseprule}{0.1pt}
\setlength{\premulticols}{1pt}
\setlength{\postmulticols}{1pt}
\setlength{\multicolsep}{1pt}
\setlength{\columnsep}{0pt}



\section{Unconstrained Optimization Algorithms}
\textbf{General Ideas}
\itemizebegin
\item Select step size $s_k$ and direction $v_k$ such that $f_0(x_k + s_k v_k)$ is substantially smaller
than $f_0(x_k)$.
\item $v_k$ is usually selected such that $\nabla f_0(x_k)^\top v_k < 0$
\item First order descent
\itemizeend
\uruler
\textbf{Armijo backtracking step size rule}
\enumbegin
\item Chosse $\alpha, \beta \in (0, 1)$, set $s = 1$
\item If $f_0(x_k + s v_k) \le f_0(x_k) + s \alpha \nabla f_0(x_k) ^ \top v_k$, then return $s_k = s$.
\item Set $s = \beta s$ and return to step 2.
\enumend
\uruler
\textbf{Newton-type methods}
\itemizebegin
\item Newton's method: $v_k = -\nabla^2 f_0(x_k)^{-1} \nabla f_0(x_k)$
\item Interpretation: Minimizing second order approximation: $\min_x f_0(x_k) + \nabla f_0(x_k)^\top x + \frac{1}{2} x^\top \nabla^2 f_0(x_k) x$
\item Quasi-Newton and variable-metric: $v_k = -H_k \nabla f_0(x_k)$
\item Combined with any step size rule
\itemizeend

% --------------------------------

\section{Constrained Optimization Algorithms}
\textbf{Methods}
\itemizebegin
\item \textbf{Projected Descent:} Use a direction $v_k$ from unconstrained algorithms and compute $x^\prime_{k+1} = x_k + s_k v_k$ and then project $x^\prime_{k+1}$ onto the feasible set.
\item Construct an unconstrained problem that approximates the original problem.
\itemizeend
\uruler
\textbf{Barrier method}
\itemizebegin
\item Consider the problem $\min_x f_0(x)$ subject to $f_i(x) \le 0$ for $i = 1...m$.
\item Define the approximate problem for some parameter $t \ge 0$: $\min_x f_0(x) + \frac{1}{t}\phi (x)$, where the barrier function $\phi (x)$ tends to $\infty$ as $x$ tends to the boundary of the feasible set from inside.
\item We can use the logarithmic barrier function: $\phi (x) = -\sum_{i=1}^{m}{log(-f_i(x))}$
\item If $f_0$, $f_i$ are convex, the approximated problem is convex
\item The KKT condition is $t \nabla f_0(x) + \sum_{i=1}^{m}{\frac{1}{-f_i(x)} \nabla f_i(x)} = 0$
\item The approximation error: $0 \le f_0(x^*(t)) - p^* \le \frac{m}{t}$
\itemizeend

\section{Machine Learning}
\textbf{Linear Model}
\itemizebegin
\item Linear model: $y(x) = w^\top \phi (x) + e(x)$, where $\phi(x)$ is a feature map and $e(x)$ is a random error
\item Least square: $\hat{w} \in \argmin_w \sum_i{(w^\top \phi(x_i) - y_i)^2}$
\item Quantile for $\alpha \in [0, 1)$: $\hat{w} \in \argmin_w \frac{1}{m(1 - \alpha)} \sum_i{max\{0, w^\top \phi(x_i) - y_i\}} - \frac{1}{m} \sum_i{w^\top \phi(x_i) - y_i}$. Still a linear program.
\itemizeend
\uruler
\textbf{Support Vector Machine}
\itemizebegin
\item Classification rule: $\hat{y}(x)=sign(\phi(x)^\top x)$
\item Convex error function(hinge): $\max \{0, 1-\alpha \}$
\item Objective: $\min_w \frac{1}{m} \sum_i{\max\{0, 1-y_i \phi(x_i)^\top w\}} $
\item We can add $L_1$ or $L_2$ norm regularization on $w$
\itemizeend
\uruler
\textbf{Logistic Regression}
\itemizebegin
\item Variant of SVM with another loss function: $\min_w \frac{1}{m} \sum_i{log[1 + exp(-y_i \phi(x_i)^\top w)]}$
\item Interpretation: $P\{Y = y | X = x\} = \frac{1}{1 + exp(-y \phi(x)^\top w)}$
\itemizeend
\columnbreak

\textbf{Fisher discrimination}
\itemizebegin
\item Idea: find a direction in $\R^n$, such that the projection of data points onto the direction results in those points with positive label are from those with negative label
\item Define $A_+ \in R^{n,m_+}$ is the matrix whose column cosists of $x_i$ such that $y_i = 1$, and $A_-$ similarly for negive labeled points. The centered(mean subtracted) data matrices are
$\widetilde{A}_{\pm} = A_{\pm} (I_{m_{\pm}} - \frac{1}{m_{\pm}}\mathbf{1} \mathbf{1}^\top)$
\item The mean-squared variation of the data projected on $u$ around its centroid is:
$u^\top M u = u^\top (\frac{1}{m_-} \widetilde{A}_- \widetilde{A}_-^\top + \frac{1}{m_+} \widetilde{A}_+ \widetilde{A}_+^\top )u$
\item The goal is: $\max_{u \ne 0} \frac{(u^\top c)^2}{u^\top M u} = \min u^\top M u$ subject to $u^\top c = 1$
\item The problem is convex, hence KKT condition implies: $u = \frac{1}{c^\top M^{-1} c} M^{-1} c$
\itemizeend
\uruler
\textbf{SPCA and NNMF}
\itemizebegin
\item The original PCA problem: $\max_z z^\top \widetilde{X}\widetilde{X}^\top z $ s.t. $\|z\|_2=1$
\item In sparse PCA, we add constraint $card(z) \le k$
\item The problem leads to the rank 1 approximation problem: $\min_{p,q} \|X - p q^\top \|_F$ s.t.
$card(p) \le k, card(q) \le h$
\item One example of rank 1 approximation is: $\min_{p \ge 0,q \ge 0} \|X - p q^\top \|_F$ s.t.
$card(p) \le k, card(q) \le h$. A standard solution is coordinate descent.
\itemizeend

\section{Finance}
\textbf{Mean-Variance (Markowitz) Models}
\itemizebegin
\item $x \in \R^n$ is the wealth allocation in n groups, $r$ is the random vector of unit wealth return for each group. $\hat{r} = E[r], \Sigma = Cov(r)$
\item Hence, $\hat{r}^\top x$ is the expected return and $x^\top \Sigma x$ is the variance of the portfolio.
\item The problems are usually maximize return with variance constraints and minimize variance, with return constraints.
\item Buget constraint: $x^\top \mathbf{1} \le b$, no short-selling: $x \ge 0$
\item Sector bounds: $\sum_{i \in S}{x_i} \le \alpha x^\top \mathbf{1}$ for some $\alpha \in (0, 1)$ and sector $S$.
\item Diversification: sum of $k$ largest $x_i$ must not exceed $\eta x^\top \mathbf{1}$ for $\eta \in (0, 1)$.
This is equivalent to $kt + s^\top \mathbf{1} \le \eta x^\top \mathbf{1}$, $s \ge 0, x - t\mathbf{1} \le s$ where $t \in \R$ and $s \in \R^n$ are additional variables.
\item The Sharpe Ratio of a portfolio is $(\hat{r}^\top x - r_f) / \sqrt{x^\top \Sigma x}$, where $r_f$ is the
return of a risk-free asset.
\item If $\hat{r}^\top x > r_f$ and $\Sigma \succ 0$, then the portfolio that maximizes SR corresponds to a
point on the efficient frontier. Max SR can be formulated as SOCP.
\itemizeend
\uruler
\textbf{Value at Risk and Conditional Value at Risk}
\itemizebegin
\item The value-of-risk of a portfolio $r^\top x$ at probability level $\alpha \in (0, 1)$ is defined as
$VaR_{\alpha}(r^\top x) = -F^{-1}(\alpha)$, where $F$ is the CDF of $r^\top x$.
\item With probability no smaller that $1-\alpha$, the investor is guaranteed to receive a return higher than $-VaR_\alpha$
\item If $r$ is Gaussian with mean $\hat{r}$ and covariance $\Sigma$, we have $VaR_{\alpha}(r^\top x) \le \gamma$ is equvalent as SOC constraint $\Phi^{-1}(1-\alpha)\|\Sigma^{\frac{1}{2}}\|_2 \le \gamma + \hat{r}^\top x$
\item The conditional value at risk is the average of outcomes that are more extreme than $VaR_\alpha$
\itemizeend


\columnbreak
% --------------------------
\textbf{Capital Asset Pricing Model}
\itemizebegin
\item The CAPM says that the expected return should be
$E[r_i] = r_f + \beta_i (E[r_m] - r_f)$, where $r_f$ is the return of a risk-free asset, $r_m$ is the random return of the market and the systematic risk $\beta_i = Cov(r_i, r_m) / Var(r_m)$
\itemizeend


\section{Epi-Spline Curve Fitting}
\textbf{Epi-splines}
\itemizebegin
\item Epi-spline: Left continuous, piece-wise polynomial
\item On the $k^{th}$ segment, $q^{k}(x) = \sum_{i=0}^{p}{a_i^k x^i}$. Hence if we split the domain into $N$ segments with degree $p$ polynomial, our total number of parameters would be $N (p+1)$
\item Continuity constraint: Let $m_1 ... m_{N-1}$ be the mesh points, we have $q^k(m_k) = q^{k+1}(m_k)$ for $k = 1...N-1$.
\item Continuous differentiability: Continuity constraint plus $(q^k(m_k))^\prime = (q^{k+1}(m_k))^\prime$ for $k = 1...N-1$.
\item Fixed values: $q^{k}(x) = s(k)$
\item Monotonicity (non-increasing): $(q^k(x))^\prime \le 0$ for $x \in (m_{k-1}, m_k)$, and
$q^k(m_k) \ge q^{k+1}(m_{k+1})$. The second requirement is satisfied with continuity.
\item Convexity: Continuity constraint and $(q^k(x))^{\prime\prime} \ge 0$, $(q^k(m_k))^\prime \le (q^{k+1}(m_{k+1}))^\prime$ for $k = 1...N-1$.
\item All constraints above are linear.
\itemizeend

\section{Optimization in Control}
\textbf{Continuous LTI Systems}
\itemizebegin
\item A continuous LTI system is described by a system ODEs: $\dot{x}(t) = A x(t) + B u(t)$, $y(t) = C x(t)$
\item The solution is: $x(t) = e^{A(t - t_0)}x(t_0) + \int_{t_0}^{t}{e^{A(t-\tau)} B u(\tau) d \tau}$,
$t \ge t_0$
\itemizeend
\uruler
\textbf{Discrete-time LTI System and Discretization}
\itemizebegin
\item We discretize the time interval $t = k\Delta$ for $k = ... -1, 0, 1, ...$ and DTLTI system can be described as $x(k+1) = A x(k) + B u(k), y(k) = C x(k)$
\item The solution is: $x(k) = A^{k-k_0} x(k_0) + \sum_{i=k_0}^{k-1}{A^{k-i-1} B u(i)}$ for $k >= k_0$.
\item Assuming constant control input during one time interval, i.e. $u(t) = u(k\Delta), \forall t \in [k\Delta, (k+1)\Delta)$, we can discretize a CTLTI system by letting $A_{\Delta}=e^{A\Delta}, B_{\Delta}=\int_0^{\Delta}{e^{A\tau}d\tau}$. The we get the system $x(k+1) = A_{\Delta} x(k) + B_{\Delta} u(k)$
\itemizeend
\uruler

\textbf{Optimization-based Control Synthesis}
\itemizebegin
\item Consider a generic DTLTI system, with a scalar input signal $u(k)$ and scalar
output signal $y(k)$ (SISO)
\item Given $x(0) = x_0$, determinte control sequence $u(k), k = 0,... T-1$ such that $x(T) = x_T$
\item $x(T) = A^\top x_0 + \sum_{i=0}^{T-1}{A^{T - i - 1} B u(i)}
= A^\top x_0 + [A^{T-1}B, A^{T-2}B ... AB, B][u(0), u(1) ... u(T - 1)]^\top
= A^\top x_0 + R_T \mu_T$ where $R_T = [A^{T-1}B ... B]$
\item Define $\xi_T = x_T - A^\top x_0$, the optimization problem becomes finding a vector $\mu_T \in \R^T$
such that $R_T \mu_T = \xi_T$.
\item We assume that $T > n$ and that $R_T$ is full row rank, then the problem admits an infinite number of possible solutions. We want to choose the one with minimum effort.
\item Mimimizing energy: $\min_{\mu_T} \|\mu_T\|_2^2$ s.t. $R_T \mu_T = \xi_T$, the solution is $R_T^\top (R_T R_T^\top)^{-1} \xi_T$
\item Minimum fuel control: $\min_{\mu_T} \|\mu_T\|_1$ s.t. $R_T \mu_T = \xi_T$. The problem can be formulated as LP.
\itemizeend

\columnbreak
\textbf{Control Synthesis for Trajectory Tracking}
\itemizebegin
\item Finding a control sequence such that the output $y(k)$ tracks as closely as possible an assigned reference trajectory $y_{ref}(k)$
\item We assume SISO, $x(0) = 0$ and $y_{ref}(0) = 0$, then we get
$x(k) = A^{k-1} B u(0) + ... + ABu(k-2) + Bu(k-1)$
\item The output sequence $y(k) = Cx(k) = CA^{k-1} B u(0) + ... + CABu(k-2) + CBu(k-1)$ for $k = 1,...,T$
\item Rewrite in matrix form, we have
\[
\begin{bmatrix}
y(1)\\
y(2)\\
\vdots \\
y(T)
\end{bmatrix}
=
\begin{bmatrix}
CB & 0 & \dots & 0 \\
CAB & CB & \dots & 0 \\
\vdots & \vdots & \ddots & \vdots \\
CA^{T-1}B & \dots & CAB & CB
\end{bmatrix}
\begin{bmatrix}
u(0)\\
u(1)\\
\vdots \\
u(T-1)
\end{bmatrix}
\]that is
$y_T = \Phi_T \mu_T$

\item Define $\mathcal{Y}_{ref} = [y_{ref}(1), \dots, y_{ref}(T)]^\top$. The problem now becomes $\min_{\mu_T} \| \Phi_T \mu_T - \mathcal{Y}_{ref} \|_2^2$
\item We can also add a $L_2$ norm or instantaneous rate of variation of $\mu_T$ to penalize energy output.
\itemizeend

\uruler
\textbf{Continuous-time Lyapunov Stability Analysis}
\itemizebegin
\item The CTLTI system $\dot{x}(t) = A x(t) + B u(t)$ is said to be (asymptotically) stable if,
for $u(t) = 0$, it holds that $\lim_{t\to\infty}x(t) = 0, \forall x(0) = x_0$
\item A necessary condition is the existance of a quadratic Lyapunov function
$V(x) = x^\top P x \succ 0$ such that $\dot{V}(x) = x^\top (A^\top P + PA) x \prec 0$. The condition is equivalent to $\exists P \succ 0$ such that $A^\top P + PA \prec 0$.
\item We can reformulate the strict LMIs into non-strict LMIs:
$\exists W \succeq I$ such that $WA^\top + AW \preceq -I$.
\item Then we can use optimization technique to find $P$: $\min_{W, v} v$ subject to $WA^\top + AW \preceq -I$, $I \preceq W \preceq vI$
\itemizeend
\textbf{Stabilizing State-feedback Design}
\itemizebegin
\item Assume that the control input takes the following state-feedback form: $u(t) = Kx(t)$ where $K$ is the state-feedback gain matrix.
\item We have the controlled system: $\dot{x}(t) = (A + BK)x(t)$
\item The stability condition is: $\exists W \succeq 0, WA^\top + AW + (KW)^\top B^\top + B(KW) \preceq -I$
\item Define $Y=KW$, the condition becomes $\exists W \succeq 0, WA^\top + AW + Y^\top B^\top + BY \preceq -I$
\item Then we can solve: $\min_{W, Y, v} v + \eta \|Y\|_2$ subject to $WA^\top + AW + Y^\top B^\top + BY \preceq -I, I \preceq W \preceq vI$
\itemizeend
\uruler
\textbf{Robust Feedback Design}
\itemizebegin
\item Consider the system $\dot{x}(t) = A x(t) + B u(t)$, where now the 
\itemizeend





\end{multicols*}
\end{document}
